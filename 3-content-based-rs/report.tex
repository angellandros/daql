\documentclass[DIN, pagenumber=false, fontsize=11pt, parskip=half]{scrartcl}

\usepackage{tikz}
\usepackage{tikz-qtree}

\usepackage[utf8]{inputenc}
\usepackage{textcomp}
\usepackage{longtable}

\usepackage{enumitem}

\usepackage[german]{babel}

\usepackage[bottom=1in,top=0.5in,left=0.7in,right=0.7in]{geometry}
\geometry{a4paper}

\usepackage{hyperref}


\setlength{\parindent}{0em}

% set section in CM
\setkomafont{section}{\normalfont\bfseries\Large}

\newcommand{\mytitle}[1]{{\noindent\Large\textbf{#1}}}
\newcommand{\mysection}[1]{\textbf{\section*{#1}}}
\newcommand{\mysubsection}[2]{\romannumeral #1) #2}

% fonts

\usepackage[T1]{fontenc}
\usepackage{tgpagella}
\usepackage[euler-digits,euler-hat-accent]{eulervm}
\usepackage{amssymb}
\usepackage{graphicx}

\newcommand{\inc}[1]{\includegraphics[scale=0.4]{fig#1b.pdf} & \includegraphics[scale=0.4]{fig#1a.pdf}}

%===================================
\begin{document}

\tikzset{every tree node/.style={minimum width=2em,draw,circle},
         blank/.style={draw=none},
         edge from parent/.style=
         {draw, edge from parent path={(\tikzparentnode) -- (\tikzchildnode)}},
         level distance=1.5cm}

\noindent\textbf{Data Analysis and Query Languages} \hfill \textbf{Albert-Ludwigs-Universität Freiburg}\\
Sommersemester 2018 \hfill Mohammad-Ali A'R\^ABI \& Youssef EL-HASSANI\\

\mytitle{~~~~3. Exercise Sheet: Content-based RS \& Similarity \hfill \today}


%===================================
\mysection{Exercise 1: Content-Based Recommender}

\begin{enumerate}[label=(\alph*)]

\item % a)
Given the vector space model $\mathcal{V} = \{\mathrm{costumes}, \mathrm{halloween}, \mathrm{recommender}, \mathrm{system}, \mbox{matrix-factorizatoin}\}$, we can represent the snippets with the vectors lying in the rows of the following table:

\begin{table}[htb]
\centering
\label{my-label}
\begin{tabular}{|l|ccccc|}
\hline
   & costumes & halloween & recommender & system & matrix-factorization \\
   \hline \hline
   $D_1$ & 1 & 1 & 0 & 0 & 0 \\
   $D_2$ & 0 & 1 & 0 & 0 & 0 \\
   $D_3$ & 1 & 0 & 0 & 0 & 0 \\
   $D_4$ & 0 & 0 & 1 & 1 & 0 \\
   $D_5$ & 0 & 0 & 1 & 1 & 1 \\
   \hline
\end{tabular}
\caption{TF representation of the snippets}
\end{table}

\end{enumerate}
\end{document}
