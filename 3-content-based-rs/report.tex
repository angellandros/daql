\documentclass[DIN, pagenumber=false, fontsize=11pt, parskip=half]{scrartcl}

\usepackage{tikz}
\usepackage{tikz-qtree}

\usepackage[utf8]{inputenc}
\usepackage{textcomp}
\usepackage{longtable}

\usepackage{amsmath}

% for alphabet items
\usepackage{enumitem}

\usepackage[bottom=1in,top=1in,left=0.7in,right=0.7in]{geometry}
\geometry{a4paper}

\usepackage{hyperref}

% fancy header
\usepackage{fancyhdr}

\pagestyle{fancy}
\fancyhf{}
\rhead{Mohammad-Ali A'R\^ABI \& Youssef EL-HASSANI}
\lhead{DAQL SS2018 :: 3. Exercise Sheet}
\rfoot{Page \thepage}

\setlength{\parindent}{0em}

% new notation

\newcommand{\prob}[1]{\mathbb{P}\left[ #1 \right]}
\newcommand{\probm}[2]{\mathbb{P}\left[ #1 ~\middle|~ #2 \right]}
\newcommand{\D}{\mathcal{D}}

% set section in CM
\setkomafont{section}{\normalfont\bfseries\Large}

\newcommand{\mytitle}[1]{{\noindent\Large\textbf{#1}}}
\newcommand{\mysection}[1]{\textbf{\section*{#1}}}
\newcommand{\mysubsection}[2]{\romannumeral #1) #2}

% fonts

\usepackage[T1]{fontenc}
\usepackage{tgpagella}
\usepackage[euler-digits,euler-hat-accent]{eulervm}
\usepackage{amssymb}
\usepackage{graphicx}

\newcommand{\inc}[1]{\includegraphics[scale=0.4]{fig#1b.pdf} & \includegraphics[scale=0.4]{fig#1a.pdf}}

%===================================
\begin{document}
\thispagestyle{empty}

\tikzset{every tree node/.style={minimum width=2em,draw,circle},
         blank/.style={draw=none},
         edge from parent/.style=
         {draw, edge from parent path={(\tikzparentnode) -- (\tikzchildnode)}},
         level distance=1.5cm}

\noindent\textbf{Data Analysis and Query Languages} \hfill \textbf{Albert-Ludwigs-Universität Freiburg}\\
Sommersemester 2018 \hfill Mohammad-Ali A'R\^ABI \& Youssef EL-HASSANI\\

\mytitle{~~~~3. Exercise Sheet: Content-based RS \& Similarity \hfill \today}


%===================================
\mysection{Exercise 1: Content-Based Recommender}

\begin{enumerate}[label=(\alph*)]

\item % a)
Given the vector space model $\mathcal{V} = \{\mathrm{costumes}, \mathrm{halloween}, \mathrm{recommender}, \mathrm{system}, \mbox{matrix-factorizatoin}\}$, we can represent the snippets with the vectors lying in the rows of the following table:

\begin{table}[htb]
\centering
\label{tab:bob-and-5}
\begin{tabular}{|l|ccccc|c|}
\hline
   & costumes & halloween & recommender & system & matrix-factorization & preference \\
   \hline \hline
   $D_1$ & 1 & 1 & 0 & 0 & 0 & 1 \\
   $D_2$ & 0 & 1 & 0 & 0 & 0 & 1 \\
   $D_3$ & 1 & 0 & 0 & 0 & 0 & n/a \\
   $D_4$ & 0 & 0 & 1 & 1 & 0 & 0 \\
   $D_5$ & 0 & 0 & 1 & 1 & 1 & n/a \\
   \hline
\end{tabular}
\caption{TF representation of the snippets, together with Bob's preference}
\end{table}

\item % b)
Denote Bob's preference by the random variable $\ell$. We are interested in calculating
\[
\probm{\ell = 1}{\D_i} = \frac{\prod_j \probm{\D_i^j}{\ell = 1}\prob{\ell = 1}}{\prob{\D_i}},
\]
with $\D_i^j$ denoting $D^j = D_i^j$ and $\D_i$ denoting $\bigwedge_j \D_i^j$.

For the case of $i=3$,
\begin{eqnarray*}
\prod_j \probm{\D_3^j}{\ell = 1} &=& \probm{D^1 = 1}{\ell = 1} \times \probm{D^2 = 0}{\ell = 1} \times \probm{D^3 = 0}{\ell = 1} \times \\
&& \probm{D^4 = 0}{\ell = 1} \times \probm{D^5 = 0}{\ell = 1} \\
&=& 0.5 \times 0 \times 1 \times 1 \times 1 \\
&=& 0.
\end{eqnarray*}

Hence $\probm{\ell = 1}{\D_3} = 0$. For the case of $i = 5$ also we have $\probm{D^2 = 0}{\ell=1} = 0$, hence $\probm{\ell = 1}{\D_5} = 0$. So, there won't be any preference between $D_3$ and $D_5$.

\item % c)

We have $D^+ = \{D_1, D_2\}$ and $D^- = \{D_4\}$. So
\begin{eqnarray*}
u_p &=& \beta \left( \frac{1}{|D^+|} \sum_{d \in D^+} d \right) - \gamma \left( \frac{1}{|D^-|} \sum_{d \in D^-} d \right) \\
&=& 0.4 \left( \begin{bmatrix} 1 \\ 1 \\ 0 \\ 0 \\ 0 \end{bmatrix} + \begin{bmatrix} 0 \\ 1 \\ 0 \\ 0 \\ 0 \end{bmatrix} \right) - 0.2 \begin{bmatrix} 0 \\ 0 \\ 1 \\ 1 \\ 0 \end{bmatrix} \\
&=& \begin{bmatrix} 0.4 & 0.8 & -0.2 & -0.2 & 0 \end{bmatrix}^T.
\end{eqnarray*}

\item
Comparing Bob's profile $u_p$ with $D_3$ and $D_5$ with cosine similarity
\[
\mathrm{sim}(A, B) = \frac{\left\langle A, B \right\rangle}{\|A\| \cdot \|B\|},
\]
we have $\mathrm{sim}(u_p, D_3) \approx 0.426$ and $\mathrm{sim}(u_p, D_5) \approx -0.246$. Hence $D_3$ is recommended to Bob.

\end{enumerate}

%===================================
\mysection{Exercise 2: Jaccard Similarity}

\begin{enumerate}[label=(\alph*)]

\item % a)
The table below contains $\mathfrak{J}$accard similarity in the upper triangle and $\mathfrak{J}$accard distance in the lower triangle.

\begin{table}[!ht]
    \centering
    \begin{tabular}{c|ccc}
        $\mathfrak{J}$ & $S_1$ & $S_2$ & $S_3$ \\
        \hline
        $S_1$ & & $1/3$ & $2/7$ \\
        $S_2$ & $2/3$ & & $1/6$ \\
        $S_3$ & $5/7$ & $5/6$
    \end{tabular}
    \caption{$\mathfrak{J}$accard similarity and distance}
    \label{tab:jaccard}
\end{table}

\item
Since we have 
\[
|A \cup B| = |A \cap B| + |A \setminus B| + |B \setminus A|,
\]
so
\[
1 - \frac{|A \cup B|}{|A \cap B|} = \frac{|A \cap B| - |A \cup B|}{|A \cap B|} = \frac{|A \setminus B| + |B \setminus A|}{|A \cap B|}.
\]
\end{enumerate}

%===================================
\mysection{Exercise 3: Similarity Join}

\begin{enumerate}[label=(\alph*)]

\item % a)

With $S_{\mathrm{stop}} = \{\mathrm{it}, \mathrm{is}, \mathrm{of}\}$, after converting all the characters to small, removing punctuation, and removing double spaces, we get
\begin{align*}
    & T_1 = \langle \mbox{\textvisiblespace payday} \rangle, \\
    & T_2 = \langle \mbox{mayday{}\textvisiblespace{}mayday} \rangle, \\
    & T_3 = \langle \mbox{day{}\textvisiblespace{}may} \rangle.
\end{align*}
The index\footnote{\url{https://gist.github.com/angellandros/25a66cee2989dd05e9f4d18aabc4afb8}} is represented in table \ref{tab:index_mayday}.

\begin{table}[!ht]
    \centering
    \begin{tabular}{|l|l|}
        \hline
        word & docs \\
        \hline \hline
        {}\textvisiblespace{}mayd & $\{ T_2 \}$ \\
        {}\textvisiblespace{}payd & $\{ T_1 \}$ \\
        ay{}\textvisiblespace{}ma & $\{ T_2, T_3 \}$ \\
        ayday & $\{ T_1, T_2 \}$ \\
        day{}\textvisiblespace{}m & $\{ T_2, T_3 \}$ \\
        mayda & $\{ T_2 \}$ \\
        payda & $\{ T_1 \}$ \\
        y{}\textvisiblespace{}may & $\{ T_2, T_3 \}$ \\
        yday{}\textvisiblespace{} & $\{ T_2 \}$ \\ \hline
    \end{tabular}
    \caption{Inverted index}
    \label{tab:index_mayday}
\end{table}

\item

For extracting similar documents, we may calculate $\mathfrak{J}$accard similarity directly, or compare the overlap count with the overlap threshold calculated with
\[
\mathrm{thr}(D, D', t) = \frac{t}{1+t} \cdot \left( \left| D \right| + \left| D' \right| \right),
\]
with $t$ be $\mathfrak{J}$accard similarity threshold, here $t = 0.25$. Representing texts as shingles, we have
\begin{align}
    & \hat{T}_1 = \{ \mbox{{}\textvisiblespace{}payd}, \mbox{ayday}, \mbox{payda} \}, \nonumber \\
    & \hat{T}_2 = \{ \mbox{{}\textvisiblespace{}mayd}, \mbox{ay{}\textvisiblespace{}ma}, \mbox{ayday}, \mbox{day{}\textvisiblespace{}m}, \mbox{mayda}, \mbox{y{}\textvisiblespace{}may}, \mbox{yday{}\textvisiblespace{}} \}, \label{eq:shingles} \\
    & \hat{T}_3 = \{ \mbox{ay{}\textvisiblespace{}ma}, \mbox{day{}\textvisiblespace{}m}, \mbox{y{}\textvisiblespace{}may} \}. \nonumber
\end{align}

The table \ref{tab:compare_d2} shows the values of $\mathrm{thr}(\hat{T}_2, \hat{T}_i, \frac{1}{4})$ for $i \in \{1, 3\}$. The $\mathfrak{J}$accard column verifies the results achieved from the overlap threshold. As $\left| \hat{T}_1 \right| = \left| \hat{T}_3 \right| = 3$ and $\left| \hat{T}_2 \right| = 7$ we have $\left| \hat{T}_2 \right| + \left| \hat{T}_i \right| = 10$ and
\[
\mathrm{thr}(\hat{T}_2, \hat{T}_i, \frac{1}{4}) = \frac{\frac{1}{4}}{\frac{5}{4}} \cdot 10 = 2,
\]
for $i \in \{1, 3\}$.

\begin{table}[!hb]
    \centering
    \begin{tabular}{|l|cc|c|}
        \hline
        doc & ovr & thr & $\mathfrak{J}$ \\ 
        \hline \hline
        $\hat{T}_1$ & 1 & 2 & $\frac{1}{9}$ \\
        $\hat{T}_3$ & 3 & 2 & $\frac{3}{7}$ \\
        \hline
    \end{tabular}
    \caption{Comparing $\hat{T}_2$ to the other documents}
    \label{tab:compare_d2}
\end{table}

One can observe that $\frac{1}{9} < \frac{1}{4}$ and $\frac{3}{7} > \frac{1}{4}$, hence $\hat{T}_3$ is related to $\hat{T}_2$, but $\hat{T}_1$ isn't.

\item

The shingles are already sorted in \eqref{eq:shingles}. With prefix length being
\[
\mathrm{pref}(D) = |D| - \left\lceil t \cdot |D| \right\rceil + 1,
\]
we have $\mathrm{pref}(\hat{T}_1) = \mathrm{pref}(\hat{T}_2) = 3$ and
\[
\mathrm{pref}(\hat{T}_2) = \left| \hat{T}_2 \right| - \left\lceil \frac{1}{4} \cdot \left| \hat{T}_2 \right| \right\rceil + 1 = 7 - \left\lceil \frac{1}{4} \cdot 7 \right\rceil + 1 = 7 - 2 + 1 = 6,
\]
hence we can ignore the last element of $\hat{T}_2$:
\begin{align*}
    & \hat{T}_1 = \{ \mbox{\underline{{}\textvisiblespace{}payd}}, \underline{\mbox{ayday}}, \underline{\mbox{payda}} \}, \\
    & \hat{T}_2 = \{ \underline{\mbox{{}\textvisiblespace{}mayd}}, \underline{\mbox{ay{}\textvisiblespace{}ma}}, \underline{\mbox{ayday}}, \underline{\mbox{day{}\textvisiblespace{}m}}, \underline{\mbox{mayda}}, \underline{\mbox{y{}\textvisiblespace{}may}}, \mbox{yday{}\textvisiblespace{}} \}, \\
    & \hat{T}_3 = \{ \underline{\mbox{ay{}\textvisiblespace{}ma}}, \underline{\mbox{day{}\textvisiblespace{}m}}, \underline{\mbox{y{}\textvisiblespace{}may}} \}.
\end{align*}

\end{enumerate}

%===================================
\mysection{Exercise 4: Min-Hashing and LSH}

\begin{enumerate}[label=(\alph*)]

\item % a)

The signature matrix is given in table \ref{tab:minhash}.

\begin{table}[!ht]
    \centering
    \begin{tabular}{ccccc}
        & $S_1$ & $S_2$ & $S_3$ & $S_4$ \\
        \hline
        $h_1$ & 2 & 1 & 1 & 2 \\
        $h_2$ & 1 & 2 & 2 & 1 \\
        $h_3$ & 1 & 2 & 4 & 1 \\
        $h_4$ & 1 & 2 & 2 & 1 \\
        $h_5$ & 1 & 2 & 2 & 1 \\
        $h_6$ & 1 & 4 & 5 & 1
    \end{tabular}
    \caption{Minhash signature matrix}
    \label{tab:minhash}
\end{table}

Moreover, the $\mathfrak{J}$accard of sets and their signatures are given in 

\begin{table}[]
    \centering
    \begin{tabular}{l||c|c|c|c}
        $\mathfrak{J}$accard similarity & 1-3 & 1-4 & 2-3 & 2-4 \\
        \hline \hline
        between sets       & 0 & 3/4 & 3/4 & 0 \\
        between signatures & 0 & 1 & 2/3 & 0
    \end{tabular}
    \caption{$\mathfrak{J}$accard similarity of sets and minhash signatures}
    \label{tab:my_label}
\end{table}

\end{enumerate}
\end{document}
