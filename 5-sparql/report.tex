\documentclass[DIN, pagenumber=false, fontsize=11pt, parskip=half]{scrartcl}

\usepackage{tikz}
\usepackage{tikz-qtree}

\usepackage[utf8]{inputenc}
\usepackage{textcomp}
\usepackage{longtable}

\usepackage{amsmath}

% for alphabet items
\usepackage{enumitem}

\usepackage[bottom=1in,top=1in,left=0.7in,right=0.7in]{geometry}
\geometry{a4paper}

\usepackage{hyperref}

% fancy header
\usepackage{fancyhdr}

\pagestyle{fancy}
\fancyhf{}
\rhead{Mohammad-Ali A'R\^ABI \& Youssef EL-HASSANI}
\lhead{DAQL SS2018 :: 5. Exercise Sheet}
\rfoot{Page \thepage}

\setlength{\parindent}{0em}

% new notation

\newcommand{\prob}[1]{\mathbb{P}\left[ #1 \right]}
\newcommand{\probm}[2]{\mathbb{P}\left[ #1 ~\middle|~ #2 \right]}
\newcommand{\D}{\mathcal{D}}

% set section in CM
\setkomafont{section}{\normalfont\bfseries\Large}

\newcommand{\mytitle}[1]{{\noindent\Large\textbf{#1}}}
\newcommand{\mysection}[1]{\textbf{\section*{#1}}}
\newcommand{\mysubsection}[2]{\romannumeral #1) #2}

% fonts

\usepackage[T1]{fontenc}
\usepackage{lmodern}
\usepackage{tgpagella}
\usepackage[euler-digits,euler-hat-accent]{eulervm}
\usepackage{amssymb}
\usepackage{graphicx}

% code hightlighting
\usepackage{listings}
\lstset{language=SQL,morekeywords={PREFIX,foaf,movies,xsd}}


\newcommand{\inc}[1]{\includegraphics[scale=0.4]{fig#1b.pdf} & \includegraphics[scale=0.4]{fig#1a.pdf}}

%===================================
\begin{document}
\thispagestyle{empty}

\tikzset{every tree node/.style={minimum width=2em,draw,circle},
         blank/.style={draw=none},
         edge from parent/.style=
         {draw, edge from parent path={(\tikzparentnode) -- (\tikzchildnode)}},
         level distance=1.5cm}

\noindent\textbf{Data Analysis and Query Languages} \hfill \textbf{Albert-Ludwigs-Universität Freiburg}\\
Sommersemester 2018 \hfill Mohammad-Ali A'R\^ABI \& Youssef EL-HASSANI\\

\mytitle{~~~~5. Exercise Sheet: SPARQL1.1 \& nSPARQL \& TriAL \hfill \today}


%===================================
\mysection{Exercise 1: Aggregations, Subqueries, Explicit Negation}

\begin{enumerate}[label=\alph*)]

\item The average of Alice's rating to the action movies is 7.5, while Bob's ratings has an average of 6.5.

\begin{lstlisting}[captionpos=b, caption=SPARQL query :: average of ratings in action genre, label=lst:sparql,
   basicstyle=\ttfamily,frame=single]
PREFIX foaf:<http://xmlns.com/foaf/0.1/>
PREFIX movies:<http://example.org/movies#>
PREFIX xsd:<http://www.w3.org/2001/XMLSchema#>

SELECT ?user
       ?genre
       (AVG(?rating) as ?mean)
WHERE
{
    ?user movies:hasRated   ?x .
    ?x    movies:hasRating  ?rating ;
          movies:ratedMovie ?m .
    ?m    movies:hasGenre   ?genre .
}
GROUP BY ?user
         ?genre
\end{lstlisting}

\begin{table}[!ht]
    \centering
    \begin{tabular}{l|cccc}
         & action & sci-fi & thriller & drama \\
        \hline
        Alice & 7.5 & 4.0 & 8.75 & 3.0 \\
        Bob & 6.5 & 4.0 & 9.0 & -
    \end{tabular}
    \caption{SPARQL query result}
    \label{tab:my_label}
\end{table}

\item
For this part, Alice does not have a similar taste to Bob.

\begin{lstlisting}[captionpos=b, caption=Finding people similar to Bob, label=lst:sparql,
   basicstyle=\ttfamily,frame=single]
PREFIX foaf:<http://xmlns.com/foaf/0.1/>
PREFIX movies:<http://example.org/movies#>
PREFIX xsd:<http://www.w3.org/2001/XMLSchema#>
SELECT ?user
       (MAX(?diff) as ?max)
WHERE {
    {
    SELECT ?user
           ?genre
           (AVG(?rating) as ?mean)
           (AVG(?bobrating) as ?bobm)
           (ABS(?mean - ?bobm) as ?diff)
    WHERE
    {
        ?user movies:hasRated   ?x .
        ?x    movies:hasRating  ?rating ;
              movies:ratedMovie ?m .
        ?m    movies:hasGenre   ?genre .
        movies:Bob movies:hasRated   ?y .
        ?y         movies:hasRating  ?bobrating ;
                   movies:ratedMovie ?bobm .
        ?bobm      movies:hasGenre   ?genre .
    }
    GROUP BY ?user
             ?genre
    }
}
GROUP BY ?user
HAVING (MAX(?diff) < 1)
\end{lstlisting}

\end{enumerate}

%===================================
\mysection{Exercise 2: nSPARQL}

\begin{enumerate}[label=\alph*)]

\item 
$ P1 = (?x, (next :: TGV | next :: Seafrance)+, Dover) $\\
$ [\![P1]\!] = \left\{ \mu| dom(\mu)=\left\{ ?x\right\} and (\mu(?x),Dover) \in [\![(next::TGV|next::Seafrance)+]\!]\right\}$\\
--\\
$[\![(next::TGV|next::Seafrance)]\!] = [\![(next::TGV]\!] \cup [\![(next::Seafrance]\!]$\\
$= (\left\{ (x,y)|(x,TGV,y)\in G\right\} \cup\left\{ (x,y)|(x,Seafrance,y)\in G\right\})$\\
$= (\left\{ (Paris, Calais),(Paris,Dijon)\right\} \cup\left\{ (Calais, Dover)\right\})$\\
$= (\left\{ (Paris, Calais),(Paris,Dijon),(Calais, Dover)\right\})$\\
--\\
$ [\![P1]\!] = \left\{ \mu| dom(\mu)=\left\{ ?x\right\} and (\mu(?x),Dover) \in [\![(next::TGV|next::Seafrance)+]\!]\right\}$\\
$ [\![P1]\!] = \left\{\left\{ ?x \Rightarrow Calais \right\},\left\{ ?x \Rightarrow Paris \right\}\right\}$\\

\item 
$ P2 = (?x, (next :: TGV | next :: Sea france)+, Dover) OPT(?x, next :: country, ?y) $\\
$ [\![P2]\!] = [\![P1]\!] OPT [\![T(?x, next :: country, ?y)]\!]$\\
$ [\![P2]\!] = [\![P1]\!]LeftOuterJoin[\![T(?x, next :: country, ?y)]\!]$\\
--\\
$[\![(?x, next :: country, ?y)]\!] = \left\{ \mu| dom(\mu)=\left\{ ?x, ?y\right\} and (\mu(?x),\mu(?y)) \in [\![(next::country)]\!]\right\}$\\
$ [\![(next::country)]\!] = (\left\{ (x,y)|(x,country,y)\in G\right\}$\\
$ [\![(next::country)]\!] = \left\{ (Paris, France)\right\}$\\
$[\![(?x, next :: country, ?y)]\!] = \left\{ \mu| dom(\mu)=\left\{ ?x, ?y\right\} and (\mu(?x),\mu(?y)) \in\left\{ (Paris, France)\right\}\right\}$\\
$ [\![(?x, next :: country, ?y)]\!] = \left\{\left\{ ?x \Rightarrow Paris \right\},\left\{ ?y \Rightarrow France \right\}\right\}$\\
--\\
$ [\![P2]\!] = [\![P1]\!]LeftOuterJoin[\![T(?x, next :: country, ?y)]\!]$\\
$ [\![P2]\!] = \left\{\left\{ ?x \Rightarrow Calais \right\},\left\{ ?x \Rightarrow Paris \right\}\right\}LeftOuterJoin\left\{\left\{ ?x \Rightarrow Paris \right\},\left\{ ?y \Rightarrow France \right\}\right\}$\\
$ [\![P2]\!] = \left\{\left\{ ?x \Rightarrow Calais \right\},\left\{\left\{ ?x \Rightarrow Paris \right\},\left\{ ?y \Rightarrow France \right\}\right\}\right\}$\\

\item 
$P3 = (?x, (next :: Seafrance | next :: NExpress)+ / self :: [next :: NExpress = self :: London]
/ (next :: Seafrance | next :: NExpress)+, ?y)$\\
--\\
$(next :: Seafrance | next :: NExpress)+ would return:\\
\left\{ (Calais, Dover),(Dover, Hastings),(Dover,London)\right\}$\\
--\\
After applying $self :: [next :: NExpress = self :: London]$ the result would be:\\
Dover because London can only be accesed through NExpress from Dover\\
--\\
After applying $(next :: Seafrance | next :: NExpress)+$ would return:\\
$\left\{(Dover, Hastings),(Dover,London)\right\}$\\
$ [\![P3]\!] = \left\{\left\{\left\{ ?x \Rightarrow Dover \right\},\left\{ ?y \Rightarrow Hastings \right\}\right\},\left\{\left\{ ?x \Rightarrow Dover \right\},\left\{ ?y \Rightarrow London \right\}\right\}\right\}$\\
\item 

$P4 = (?x, (next :: [(next :: sp) / self :: transport])+, ?y)$\\
--\\
$next :: [(next :: sp) / self :: transport]$ would return subjects that have predicates\\
that are subclasses of transport. The results would be:\\
$(Paris,Calais), (Paris,Dijon), (Calais, Dover), (Dover,Hastings), (Dover,London)$\\
Since we apply the expression above more than once we will end up with the following results:\\
$(Paris,Calais), (Paris,Dijon), (Paris,Dover), (Paris,Hastings), (Paris, London), (Calais, Dover),$\\
$(Calais, Hastings), (Calais,London), (Dover,Hastings), (Dover,London)$\\
Thus the result for P4 is the following:
$ [\![P4]\!] = \left\{\left\{\left\{ ?x \Rightarrow Paris \right\},\left\{ ?y \Rightarrow Calais \right\}\right\},\left\{\left\{ ?x \Rightarrow Paris \right\},\left\{ ?y \Rightarrow Dijon \right\}\right\}...\right\}$\\
The etc reffers to the order refered above, it will basically have a mapping for every possible route \\
using any possible transportation.\\

\item 
$P5 = (?x, ((trans(train)|trans(ferry))+ / self :: [trans(type) = sel f :: costal_city]), ?y)$\\
\item 
trans can be applied here in order to check if the object is a $costal\_city$.
\item 
trans can be applied in the same way as in f. However, the object here will either be a city or a $costal\_city$.
\end{enumerate}
%===================================
\mysection{Exercise 3: TriAL}

\begin{enumerate}[label=\alph*)]

\item
The result of the right Kleene closure is:

\begin{table}[!ht]
    \centering
    \begin{tabular}{c|c|c}
        \texttt{St.~Andrew} & \texttt{Bus Op 1} & \texttt{London} \\
        \texttt{Edinburgh} & \texttt{Train Op 1} & \texttt{Brussels} \\
        \texttt{Train Op 1} & \texttt{part\_of} & \texttt{NatExpress} \\
        \texttt{St.~Andrew} & \texttt{Bus Op 1} & \texttt{Brussels} \\
    \end{tabular}
    \caption{Kleene right closure}
    \label{tab:rcl}
\end{table}

\item
The result of the left Kleene closure is:

\begin{table}[!ht]
    \centering
    \begin{tabular}{c|c|c}
        \texttt{St.~Andrew} & \texttt{part\_of} & \texttt{NatExpress} \\
        \texttt{Edinburgh} & \texttt{part\_of} & \texttt{EastCoast} \\
        \texttt{London} & \texttt{part\_of} & \texttt{EuroStar} \\
    \end{tabular}
    \caption{Kleene left closure}
    \label{tab:lcl}
\end{table}


\end{enumerate}



\vspace{2em}
Mohammad-Ali A'R\^ABI and Youssef EL-HASSANI \\
Freiburg im Breisgau, 05.07.2018

\end{document}

