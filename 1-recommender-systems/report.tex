\documentclass[DIN, pagenumber=false, fontsize=12pt, parskip=half]{scrartcl}

\usepackage{tikz}
\usepackage{tikz-qtree}

\usepackage[utf8]{inputenc}
\usepackage{textcomp}
\usepackage{longtable}

\usepackage[bottom=1in,top=0.5in,left=0.7in,right=0.7in]{geometry}
\geometry{a4paper}

\usepackage{hyperref}


\setlength{\parindent}{0em}

% set section in CM
\setkomafont{section}{\normalfont\bfseries\Large}

\newcommand{\mytitle}[1]{{\noindent\Large\textbf{#1}}}
\newcommand{\mysection}[1]{\textbf{\section*{#1}}}
\newcommand{\mysubsection}[2]{\romannumeral #1) #2}

% fonts

\usepackage[T1]{fontenc}
\usepackage{tgpagella}
%\usepackage[euler-digits,euler-hat-accent]{eulervm}
\usepackage{amssymb}
\usepackage{graphicx}

\newcommand{\inc}[1]{\includegraphics[scale=0.4]{fig#1b.pdf} & \includegraphics[scale=0.4]{fig#1a.pdf}}

%===================================
\begin{document}

\tikzset{every tree node/.style={minimum width=2em,draw,circle},
         blank/.style={draw=none},
         edge from parent/.style=
         {draw, edge from parent path={(\tikzparentnode) -- (\tikzchildnode)}},
         level distance=1.5cm}

\noindent\textbf{Data Analysis and Query Languages} \hfill \textbf{Albert-Ludwigs-Universität Freiburg}\\
Sommersemester 2018 \hfill Mohammad-Ali A'R\^ABI \& Youssef El Hassani\\

\mytitle{1. Exercise Sheet: Recommender Systems \hfill \today}


%===================================
\mysection{Exercise 1: Collaborative-filtering Recommender Systems}

a) Given the following table:

\begin{table}[htb]
\centering

\label{my-label}
\begin{tabular}{lcccccc}
\hline
       & Item 1 & Item 2 & Item 3 & Item 4 & Item 5 & Item 6 \\ \hline
Alice  & 5      & 3      & 4      & ?      & 1      & ?      \\
User 1 & 3      & 1      & 2      & 3      & 1      & 3      \\
User 2 & 4      & 3      & 4      & 3      & 1      & 5      \\
User 3 & 3      & 3      & 1      & 5      & 1      & 4      \\
User 4 & 1      & 5      & 5      & 2      & 1      & 1     \\ \hline
\end{tabular}
\end{table}

 The cosine similarities between Alice and all the other users are the following: \\
\begin{center}
$	sim ( Alice, User1) = 0.9762 $ \\
$	sim ( Alice, User2) = 0.9939 $ \\
$	sim ( Alice, User3) = 0.9080 $ \\
$	sim ( Alice, User4) = 0.7962 $ \\
\end{center}
The predictions computed from the prediction function are the following: \\

$ pred ( Alice, Item 4) = \frac{13}{4} + \frac{1}{\left | 0.9762 + 0.9939 \right |} \cdot \left ( 0.9939 \cdot \left ( 3-\frac{20}{6} \right ) + 0.9762 \cdot \left (3-\frac{13}{6}\right) \right ) $  \\
$ pred ( Alice, Item 4) = 3.4947$ \\

$ pred ( Alice, Item 6) = \frac{13}{4} + \frac{1}{\left | 0.9762 + 0.9939 \right |} \cdot \left ( 0.9939 \cdot \left ( 5-\frac{20}{6} \right ) + 0.9762 \cdot \left (3-\frac{13}{6}\right) \right ) $  \\
$ pred ( Alice, Item 6) = 4.5037$ \\

%===================================
\mysection{Exercise 2: Similarity metrics}

a) $ Proof :$

We have the following cosine similarity formula:
\begin{center}
$ sim ( X, Y ) = \frac{\left | X \right | \cdot \left | Y \right |}{\sqrt{X^{2}} \cdot \sqrt{Y^{2}}}$ \\
\end{center}
We now replace X by bX and Y by dY:
\begin{center}
$ sim ( X, Y ) =\frac{\left | bX \right | \cdot \left | dY \right |}{\sqrt{(bX)^{2}} \cdot \sqrt{(dY)^{2}}}$ \\
\end{center}
By equivalence: 
\begin{center}
$ sim ( X, Y ) =\frac{\left | b \right |\cdot \left | X \right | \cdot \left | d \right |\cdot\left | Y \right |}{\sqrt{(b)^{2}} \cdot \sqrt{(X)^{2}} \cdot \sqrt{(d)^{2}} \cdot \sqrt{(Y)^{2}}}$ \\
$ sim ( X, Y ) =\frac{\left | b \right |\cdot \left | X \right | \cdot \left | d \right |\cdot\left | Y \right |}{\left | b \right | \cdot \sqrt{(X)^{2}} \cdot \left | d \right | \cdot \sqrt{(Y)^{2}}}$\\
\end{center}
By eliminating b and d we get:
\begin{center}
$ sim ( X, Y ) = \frac{\left | X \right | \cdot \left | Y \right |}{\sqrt{X^{2}} \cdot \sqrt{Y^{2}}}$ \\
\end{center}
Thus, the cosine similarity is invariant to the scale of variables.

b) $ Proof :$
We have the following Pearson similarity formula:
\begin{center}
$ sim ( X, Y ) = \frac{\sum \left ( X -\bar{X}\right ) \cdot \left ( Y -\bar{Y}\right )}{\sqrt{\sum \left ( X-\bar{X} \right )^{2} \sum \left ( Y-\bar{Y}\right )^{2} }}$ \\
We then replace \\
$ a+bX for X$ \\
$ c+dY for Y$\\
\end{center}
\begin{center}
Remark that the previous mean values will become:\\
$a+b\bar{X} for \bar{X}$ \\
$ c+d\bar{X} for \bar{X}$ \\
\end{center}
By equivalence we get: \\

\begin{center}
$sim ( X, Y ) = \frac{\sum \left ( a + bX - a - b\bar{X}\right ) \cdot \left ( c + dY - c -d\bar{Y}\right )}{\sqrt{\sum \left ( a + bX- a - b\bar{X} \right )^{2} \sum \left ( c + dY-c-d\bar{Y}\right )^{2} }}$\\
$sim( X, Y ) = \frac{\sum \left ( bX - b\bar{X}\right ) \cdot \left ( dY-d\bar{Y}\right )}{\sqrt{\sum \left (bX- b\bar{X} \right )^{2} \sum \left ( dY-d\bar{Y}\right )^{2} }}$\\
$sim( X, Y ) = \frac{\sum \left ( bd \right )\cdot \left ( X - \bar{X}\right ) \cdot \left ( Y-\bar{Y}\right )}{\sqrt{\sum \left ( b \right )^{2}\cdot\left (X- \bar{X} \right )^{2} \sum \left( d \right )^{2}\cdot  \left ( Y-\bar{Y}\right )^{2} }}$\\
\end{center}

b and d are constant thefore: \\
\begin{center}
$sim ( X, Y ) =\frac{\left ( bd \right )\cdot \sum \left ( X - \bar{X}\right ) \cdot \left ( Y-\bar{Y}\right )}{\sqrt{\left ( bd \right )^{2}\cdot\sum \left (X- \bar{X} \right )^{2} \sum  \left ( Y-\bar{Y}\right )^{2} }}$\\
$sim( X, Y ) = \frac{\left ( bd \right )\cdot\sum \left ( X - \bar{X}\right ) \cdot \left ( Y-\bar{Y}\right )}{\left ( bd \right )\cdot\sqrt{\sum \left (X- \bar{X} \right )^{2} \sum  \left ( Y-\bar{Y}\right )^{2} }}$\\
$ sim ( X, Y ) = \frac{\sum \left ( X -\bar{X}\right ) \cdot \left ( Y -\bar{Y}\right )}{\sqrt{\sum \left ( X-\bar{X} \right )^{2} \sum \left ( Y-\bar{Y}\right )^{2} }}$ \\
\end{center}
Thus, the pearson similarity is invariant to separate changes of location and scale of the variables.


%===================================
\mysection{Exercise 3: Recommendations based on association rules}

The mean adjusted rating matrix R` from R:

\begin{table}[htb]
\centering
\label{my-label}
\begin{tabular}{lccccccc}
\hline
       & Item 1 & Item 2 & Item 3 & Item 4 & Item 5 & Item 6 & mean \\ \hline
Alice  & 1.75   & -0.25  & 0.75   & ?      & -2.25  & ?      & 3.25 \\
User 1 & 0.84   & -1.16  & -0.16  & 0.84   & -1.16  & 0.84   & 2.16 \\
User 2 & 0.66   & -0.33  & 0.66   & -0.33  & -2.33  & 1.66   & 3.33 \\
User 3 & 0.17   & 0.17   & -1.83  & 2.17   & -1.83  & 1.17   & 2.83 \\
User 4 & -1.5   & 2.5    & 2.5    & -0.5   & -1.5   & -1.5   & 2.5 \\ \hline
\end{tabular}
\end{table}

The transformed mean-adjusted utility matrix:\\
\begin{table}[htb]
\centering
\caption{My caption}
\label{my-label}
\begin{tabular}{lcccccc}
\hline
       & Item 1 & Item 2 & Item 3 & Item 4 & Item 5 & Item 6 \\ \hline
Alice  & 1      & 0      & 1      & ?      & 0      & ?      \\
User 1 & 1      & 0      & 0      & 1      & 0      & 1      \\
User 2 & 1      & 0      & 1      & 0      & 0      & 1      \\
User 3 & 1      & 1      & 0      & 1      & 0      & 1      \\
User 4 & 0      & 1      & 1      & 0      & 0      & 0     \\ \hline
\end{tabular}
\end{table}

Support Calculations:\\
\begin{center}
$	support ( Item 1 ) = \frac {3}{4}$\\ 
$	support ( Item 2 ) = \frac {1}{2}$\\
$	support ( Item 3 ) = \frac {1}{2} $\\
$	support ( Item 4 ) = \frac {1}{2}$\\
$	support ( Item 5 ) = 0$\\
$	support ( Item 6 ) = \frac {3}{4}$\\
$	support ( Item 1 , Item 4) = \frac {1}{2}$\\
$	support ( Item 1 , Item 6) =\frac {3}{4} $\\
$	support ( Item 1 , Item 4, team 6) =\frac {1}{2} $\\
\end{center}
Confidence Calculations:\\
\begin{center}
$	confidence (Item 1\Rightarrow Item 4) = \frac {support ( Item 1 , Item 4)}{support ( Item 1)} = \frac {\frac{1}{2}}{\frac {3}{4}}=\frac {2}{3}$\\
$	confidence (Item 1\Rightarrow Item 6) = \frac {support ( Item 1 , Item 6)}{support ( Item 1)} = \frac {\frac{3}{4}}{\frac {3}{4}}=1$\\
$	confidence (Item 1, Item 4 \Rightarrow Item 6) = \frac {support ( Item 1 , Item 4, Item 6)}{support ( Item 1 , Item 4)} = \frac {\frac{1}{2}}{\frac {1}{2}}=\frac {2}{3}$\\
$	confidence (Item 1, Item 6\Rightarrow Item 4) = \frac {support ( Item 1 , Item 6, Item 4)}{support ( Item 1 , Item 6)} = \frac {\frac{1}{2}}{\frac {3}{4}}=\frac {2}{3}$\\
\end{center}

Frequent Item sets: ( all sets greater or equal to the support treshhold) \\
\begin{center}
$	\left [ Item 1 \right ], \left [ Item 2 \right ], \left [ Item 3 \right ], \left [ Item 4 \right ], \left [ Item 6 \right ], \left [ Item 1, Item 4 \right ], \left [ Item 1, Item 6 \right ], \left [ Item 1, Item 4, Item 6 \right ]$\\
\end{center}
Inference Rules: ( all inference rules greater or equal to the confidence treshhold) \\
\begin{center}
$	\left [ Item 1\Rightarrow Item 4 \right ],\left [ Item 1\Rightarrow Item 6 \right ],\left [ Item 1,Item 4\Rightarrow Item 6 \right ],\left [ Item 1,Item 6\Rightarrow Item 4 \right ]$\\
\end{center}
This two association rules will most likely generate a recommendation because of their high confidence value: \\
\begin{center}
$	\left [ Item 1\Rightarrow Item 6 \right ],\left [ Item 1,Item 6\Rightarrow Item 4 \right ]$\\
\end{center}
The system will use the first inference rule in order to recommend Item 6 because its confidence has the highest confidence value given the initial state. Then the system will most likely recommend item 4 as a second step because it has the highest confidence given the state where the system already recommended item 6 for Alice.\\


%===================================
\mysection{Exercise 4: Recommender systems, Probabilistic Approach}

a) Calculation of priors for Item 2 and Item 4\\

\begin{table}[htb]
\centering
\label{my-label}
\begin{tabular}{lll}
\hline
  & Item 2             & Item 4            \\ \hline
1 & P(Item2 = 1) = 1/4 & P(Item4= 1) = 0   \\
2 & P(Item2 = 2) = 0   & P(Item4= 2) = 1/4 \\
3 & P(Item2 = 3) = 1/2 & P(Item4= 3) = 1/2 \\
4 & P(Item2 = 4) = 0   & P(Item4= 4) = 0   \\
5 & P(Item2 = 5) = 1/4 & P(Item4= 5) = 1/4 \\ \hline
\end{tabular}
\end{table}
b) Calculation of class-conditional probabilities for Alice`s ratings X={ Item1 = 5, Item 3= 4}:\\

\begin{table}[htb]
\centering
\label{my-label}
\resizebox{\textwidth}{!}{
\begin{tabular}{cccccc}
\hline
  & 1                             & 2                             & 3                               & 4                             & 5                             \\ \hline
Item1 & P(Item1 = 5 | Item2 = 1) = 0  & P(Item1 = 5 | Item2 = 2) = 0  & P(Item1 = 5 | Item2 = 3) = 1/2  & P(Item1 = 5 | Item2 = 4) = 0  & P(Item1 = 5 | Item2 = 5) = 0  \\
Item3 & P(Item3 = 4 |  Item2 = 1) = 0 & P(Item3 = 4 |  Item2 = 2) = 0 & P(Item3 = 4 |  Item2 = 3) = 1/2 & P(Item3 = 4 |  Item2 = 4) = 0 & P(Item3 = 4 |  Item2 = 5) = 0 \\ \hline
\end{tabular}}
\end{table}

\begin{table}[htbp]
\centering
\label{my-label}
\resizebox{\textwidth}{!}{
\begin{tabular}{cccccc}
\hline
  & 1                             & 2                             & 3                               & 4                             & 5                              \\ \hline
Item1 & P(Item1 = 5 | Item4 = 1) = 0  & P(Item1 = 5 | Item4 = 2) = 0  & P(Item1 = 5 | Item4 = 3) = 0    & P(Item1 = 5 | Item4 = 4) = 0  & P(Item1 = 5 | Item4 = 5) = 1/2 \\
Item3 & P(Item3 = 4 |  Item4 = 1) = 0 & P(Item3 = 4 |  Item4 = 2) = 0 & P(Item3 = 4 |  Item4 = 3) = 1/2 & P(Item3 = 4 |  Item4 = 4) = 0 & P(Item3 = 4 |  Item4 = 5) = 0  \\ \hline
\end{tabular}}
\end{table}
\newpage
b) Posterior probabilities of Alice`s ratings:\\

\begin{table}[htb]
\centering
\label{my-label}
\resizebox{\textwidth}{!}{
\begin{tabular}{cccccc}
\hline
 & 1 & 2 & 3 & 4 & 5 \\ \hline
Item2 & P(Item2 = 1 | X) = 0 & P(Item2 = 2 | X) = 0 & P(Item2 = 3 | X) = 1/8 & P(Item2 = 4 | X) = 0 & P(Item2 = 5 | X) = 0 \\
Item4 & P(Item4 = 1 | X) = 0 & P(Item4 = 2 | X) = 0 & P(Item4 = 3 | X) = 0 & P(Item4 = 4 | X) = 0 & P(Item4 = 5 | X) = 0 \\ \hline
\end{tabular}}
\end{table}

d) Posterior probabilities of Alice`s ratings:\\
The only value we could derive is 1/8 thefore we would recommend Item 2 and assign it a rating of 3.
\end{document}
